\documentclass{article}

\usepackage[a4paper]{geometry}
\geometry{left=2.5cm,right=2.5cm,top=3cm,bottom=3cm}
\usepackage[UTF8,space,hyperref]{ctex}
\usepackage{amsmath, amsthm, amssymb, bm, color, framed, graphicx, hyperref, mathrsfs, physics}
\hypersetup{hidelinks,colorlinks=true,allcolors=black,pdfstartview=Fit,breaklinks=true}

\title{Project 5 实验报告}
\author{钱翰林 周远扬 石曜铭}
\date{12/11/2025}

\begin{document}
\maketitle

\section{小组分工情况说明}
\begin{itemize}
        \item 周远扬: 设计类SRAM-AXI转接桥
        \item 石曜铭: 设计类SRAM交互状态机,优化流水线时序,调试
        \item 钱翰林: 调试,报告撰写
\end{itemize}

\section{处理器结构设计图}
\begin{figure}[htbp]
        \centering
        \includegraphics[width=16cm]{pics/graph.png}
\end{figure}


\section{主要设计点}

\subsection{类SRAM总线时序设计}
在exp14中,我们对原先的流水级时序进行了较大的修改。首先,流水级间信号的传递改成了用wire传向下一级,接收时用reg打一拍;其次,读寄存器时的load-block条件从暴力记录上一条指令以及pc,改为添加从EX和MEM模块到ID模块的wire,传递当前EX和MEM中是否是load,以及写的寄存器编号,这样只需在ID判断是否有冲突即可;类似的,csr指令的数据相关问题只需要分别从EX和MEM向ID输出当前指令是否为csr,以及影响的寄存器编号。这样逻辑变得清晰,时序更加规整。

在此基础上,我们按照两次握手初步完成IF级和MEM级的设计。接下来的问题是处理flush,其中MEM相对简单,因为WB流水级发出flush信号时,下一条指令必然刚进入MEM级,因此强制回到初始状态init即可废止当前指令。对于IF级,需要额外添加lock\_addr和lock\_data两个状态,分别对应于等待addr和data时收到flush信号,在这里处理被flush掉的取指请求。

\subsection{类SRAM-AXI转接桥设计}

在bridge.v中实现。主要部分是维护读请求、读响应、写请求、写响应四个状态机,并根据状态机的状态设置相应的AXI握手信号(arvalid、rready、awvalid、wvalid、bready),最后将得到的数据保存在rdata\_buffer返回,并设置addr\_ok和data\_ok信号。

\section{调试}

\subsection{类SRAM随机访问延迟验证出错}

ID、EX、MEM的状态机没有在flush时返回init,修改后可以通过各种随机访问延迟验证。

\subsection{AXI同时收到两个读请求时读响应状态机出错}
一开始设计的读响应状态机只在读请求握手成功(arvalid \& arready)时才会拉高rready。但同时处理两个读请求时,可能出现前一个读响应握手尚未完成,后一个读请求握手就已完成的情况。这样会导致第一次读响应握手完成后,状态机不会进行第二次读响应握手。  

解决方案:添加r\_cnt寄存器,用于记录完成读请求握手且未完成读响应握手的请求个数。当r\_cnt非零时,读响应状态机也会切换状态拉高rready。


\subsection{inst\_sram\_data\_ok 时序不当,导致取指错误}
inst\_sram\_data\_ok信号拉高时需要同时满足rid为0且读响应握手完成,但是rid只在rvalid拉高时才为有效值,所以直接使用rid信号会导致时序错误。

解决方案:添加rid\_reg寄存器,用于记录完成读响应握手时的rid,替换原来data\_ok里的rid。



\end{document}