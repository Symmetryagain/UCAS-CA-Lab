\documentclass{article}

\usepackage[a4paper]{geometry}
\geometry{left=2.5cm,right=2.5cm,top=3cm,bottom=3cm}
\usepackage[UTF8,space,hyperref]{ctex}
\usepackage{amsmath, amsthm, amssymb, bm, color, framed, graphicx, hyperref, mathrsfs, physics}
\hypersetup{hidelinks,colorlinks=true,allcolors=black,pdfstartview=Fit,breaklinks=true}

\title{Project 7 实验报告}
\author{钱翰林 周远扬 石曜铭}
\date{1/11/2026}

\begin{document}
\maketitle

\section{小组分工情况说明}
\begin{itemize}
        \item 周远扬: 设计cache模块,修改转接桥,报告撰写
        \item 石曜铭: 将cache集成到cpu,调试
        \item 钱翰林: 添加cacop指令,报告撰写
\end{itemize}

\section{处理器结构设计图}
\begin{figure}[htbp]
        \centering
        \includegraphics[width=15cm]{pics/graph.png}
\end{figure}


\section{主要设计点}

\subsection{Cache模块设计}
Cache需要处理读写操作,并且会有是否命中问题,因此依照着状态机的节奏,处理一次操作先需要check,检查命中情况,若未命中再向内存交互,之后replace并refill,脏块写回与新数据写入,并完成一整次的操作。
对于写情况,另有一个状态机控制单独写,主状态机认为传给写状态机之后写操作即完成。

need\_pause 包含两种情况:(1) 端口冲突:WriteBuffer 正在向一Bank 写,而新来的读请求也要读该Bank。
(2) 数据相关:Lookup 阶段是Store,而新来的读请求访问同一地址(RAW)。

\subsection{修改转接桥}

为了集成Cache,需要bridge支持burst传输,所以需要改动状态机,使得读写可以多拍延续,并且考虑last信号。

对于读写,可能是一拍或四拍,所以要考虑last等信号与实际传输过程,与Cache交互依赖状态机即可,与内存交互,就要自行计数。
\subsection{在CPU中集成Cache}

修改转接桥与top模块的相关接口,将ICache和DCache实例化。

在实际将Cache集成进cpu的过程中,由于随机延迟的存在,需要bridge和Cache之间多次握手,防止错过信号。
比如bridge中控制着传入Cache的rd\_rdy等握手信号,这些信号又控制着Cache主状态机,所以处理不当就会错过,需要一步步握手,控制信号保持,以不错漏信号处理

而且同时集成ICache,DCache,就会导致读操作冲突,因为两个读端口缩为一个,这里就需要重新构建bridge中的ar状态机来完成控制,状态分别是init,wait,req,初始为init,wait负责在rdata时等待阻塞结束,req负责rinst和wait之后的rdata,发出握手
\subsection{添加CACOP指令}

Cacop指令在EX级发出,Cache收到cacop操作的相关信号后,Hit判断可以复用LookUp访问的Tag读出和比较部分,Cache行中V的修改可以复用Refill访问的数据通路,写回内存可以复用Replace访问的数据通路。

需要注意的是hit\_invalidate进行虚实地址转换时会用到两个mmu,需要修改原先的数据通路,让EX级接收两个mmu转换的物理地址以及产生的异常信号。对于inst\_mmu,Cacop指令的优先级高于IF级的取指,这样导致IF级可能出现PC错误,因此在执行完Cacop指令后会flush进行指令重取,也可以避免相关的资源冲突。


\section{调试}

\subsection{cacop指令时序问题}

cacop\_en拉高时没有及时更新tagv\_addr和tagv\_en,而用的是慢了一拍的reg\_cacop\_en,导致cacop指令的实现时序与cache状态机的状态不符。

\subsection{I,D轮流读时,被need\_wait覆盖}

Cache中的need\_wait,指写的同时需要读,那么读等待,由于延迟随机,新的读请求被need\_wait覆盖掉,
其实是ar状态机的握手信号问题,wait时不该继续拉高,所以做出上面的改动,把wait时单独处理,不要握手以防丢掉信号

\subsection{脏块回写时内容错误}

写缺失时,要写的数据为内存旧数据叠加输入新数据,不能直接靠wstrb控制使能解决,因为wstrb为0就不再写入
所以应该先得到要写入的叠加数据,再全部写入

\subsection{wlast返回错误}

因为写操作,考虑到burst,有一拍或四拍,所以wlast应该在两种情况下都能正确拉高,依靠bridge内部的写计数完成

\end{document}