\documentclass{article}

\usepackage[a4paper]{geometry}
\geometry{left=2.5cm,right=2.5cm,top=3cm,bottom=3cm}
\usepackage[UTF8,space,hyperref]{ctex}
\usepackage{amsmath, amsthm, amssymb, bm, color, framed, graphicx, hyperref, mathrsfs, physics}
\hypersetup{hidelinks,colorlinks=true,allcolors=black,pdfstartview=Fit,breaklinks=true}

\title{Project 2 实验报告}
\author{石曜铭 2023K8009929037}
\date{10/14/2025}

\begin{document}
\maketitle

\section{处理器结构设计图}
\begin{figure}[htbp]
        \centering
        \includegraphics[width=16cm]{pics/graph.png}
\end{figure}

\section{主要设计点}

\subsection{握手信号设计}

\begin{itemize}
        \item $\texttt{valid}$: 在每个模块内部维护,表示该模块该周期是否有效,即未被flush掉。
        \item $\texttt{readygo}$: 拉高表示该模块已经做好准备将数据传到下个阶段。
        \item $\texttt{allowin}$: 拉高表示该模块下个周期可以接受来自上个阶段的数据。
\end{itemize}

$\texttt{allowin}$ 表达式:$\texttt{allowin} = (!\texttt{valid}) | (\texttt{readygo} \& \texttt{next\_allowin})$ 。
这里第一项表示如果当前状态无效,那么可以接受上个阶段的数据以挤掉气泡;第二项表示如果该阶段数据即将进入下个阶段,那么也可以接受上个阶段的数据。

\subsection{与内存交互的时序设计}
本实验中的 SRAM 是同步写同步读的,意味着与内存的交互都需要经过一个周期才会真正结束。
具体体现在取指阶段和 load 指令,都需要等待一个周期才能得到结果。虽然可以通过将数据接收放在下个阶段来解决,但这样难以扩展到多周期等待的内存交互。

于是,对于 load 指令,我选择模拟一个内存握手信号,恒置为 $1$,这样就会自然有一个周期的等待。

在 IF 阶段,因为指令是同步读的,所以每次输出的 PC 实际上是 nextPC,这样下个周期自然得到了与当前真正 PC 对应的指令。

\subsection{数据冲突:数据前递}
数据冲突本质上是当前指令是寄存器写,在还没有到 WB 阶段时,另一条指令的 ID 阶段恰好需要读这个寄存器。

数据前递用来解决写寄存器的数据在 EX 产生的情况。
只需要分别在 EX 阶段和 MEM 阶段输出已经确定的寄存器写的值和地址,接到 ID 阶段,那么在 ID 阶段就可以直接得到前两条指令的寄存器写信息,跟当前需要读的寄存器做比较,讨论情况即可。

寄存器的代码也要做一个改动,即如果异步读的地址恰好等于写地址,就直接把写数据返回。这样就可以对每条指令只往后考虑两条指令的数据前递。

\subsection{数据冲突:Load-Use 阻塞}
如果写寄存器的数据在 MEM 阶段才产生,那么就无法通过数据前递来解决冲突。在 ID 阶段,当检测到一条指令的上一条指令是 load,且写寄存器的地址恰好被读,那么需要阻塞该阶段直到上一条指令的 MEM 阶段完成,
并再开一个信号用来传递内存读出的数据。

\subsection{控制冒险}
如果一条指令在 ID 阶段被发现不应该执行,那么应该把 $\texttt{valid}$ 置为 $0$,表示该指令的数据失效。
相应地,为了保证被失效的指令不会对控制信号产生影响,应该把所有的控制信号都和 $\texttt{valid}$ 取与,这样可以避免控制冒险,例如避免被跳过的 branch 指令还会影响 PC 跳转的情况。

\end{document}