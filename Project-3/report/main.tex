\documentclass{article}

\usepackage[a4paper]{geometry}
\geometry{left=2.5cm,right=2.5cm,top=3cm,bottom=3cm}
\usepackage[UTF8,space,hyperref]{ctex}
\usepackage{amsmath, amsthm, amssymb, bm, color, framed, graphicx, hyperref, mathrsfs, physics}
\hypersetup{hidelinks,colorlinks=true,allcolors=black,pdfstartview=Fit,breaklinks=true}

\title{Project 3 实验报告}
\author{钱翰林 周远扬 石曜铭}
\date{10/28/2025}

\begin{document}
\maketitle

\section{小组分工情况说明}
\begin{itemize}
        \item 钱翰林:IP 核接口,访存指令实现
        \item 周远扬:运算指令、分支指令实现
        \item 石曜铭:调整流水线握手信号,报告撰写
\end{itemize}

\section{处理器结构设计图}
与 exp9 一致。
\begin{figure}[htbp]
        \centering
        \includegraphics[width=16cm]{pics/graph.png}
\end{figure}


\section{主要设计点}

\subsection{EX 多周期的状态设计}

由于除法器的存在,为了满足时序要求,需要将 EX 阶段改为多周期。
EX 阶段有 $4$ 个状态,分别是 $\texttt{init, wait\_src\_ready, wait\_res\_valid, readygo}$ 。
初始状态为 $\texttt{init}$,当检测到指令不需要除法时,则直接转到 $\texttt{readygo}$ 状态。
如果指令需要除法,则要先后经过两个等待状态,得到结果后在进入 $\texttt{readygo}$ 状态。

注意这里因为 EX 阶段不再是单周期,连带着 MEM 阶段的一些时序也需要修改。
原先的时序是默认 EX 阶段只有一个周期,所以修改 EX 之后会导致时序错乱而在一些测试点上保持阻塞。

\subsection{除法 IP 核的生成和接入}
本实验中采用直接生成除法 IP 核的做法。在 Vivado 中分别生成有符号和无符号除法 IP 核后,接入 EX 阶段的握手信号和数据信号。
由于除法 IP 核中有两个源操作数的 $\texttt{ready}$ 信号,实际应当对两个信号取与,才能作为源操作数的 $\texttt{ready}$ 信号。

\subsection{运算和分支指令的实现}
这两类指令的实现都在 ID 阶段。在译码时,如果检测到运算指令,则传递对应的 $\texttt{ALUop}$。
如果检测到分支指令,则判断条件之后决定是否修改 $\texttt{PC}$。

\subsection{访存指令的实现}
为了添加更多的访存指令,从 ID 向 EX 和 MEM 多传递了不同位数的访存信号;在 MEM 阶段,根据访存的位数以及内存对齐情况返回不同的结果。
相应地,阶段之间传递数据的位宽也进行了修改。

\end{document}