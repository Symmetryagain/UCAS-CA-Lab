\documentclass{article}

\usepackage[a4paper]{geometry}
\geometry{left=2.5cm,right=2.5cm,top=3cm,bottom=3cm}
\usepackage[UTF8,space,hyperref]{ctex}
\usepackage{amsmath, amsthm, amssymb, bm, color, framed, graphicx, hyperref, mathrsfs, physics}
\hypersetup{hidelinks,colorlinks=true,allcolors=black,pdfstartview=Fit,breaklinks=true}

\title{Project 6 实验报告}
\author{钱翰林 周远扬 石曜铭}
\date{12/16/2025}

\begin{document}
\maketitle

\section{小组分工情况说明}
\begin{itemize}
        \item 周远扬: 调试
        \item 石曜铭: 设计tlb模块,添加tlb相关指令,调试
        \item 钱翰林: 添加相关的CSR,报告撰写
\end{itemize}

\section{处理器结构设计图}
\begin{figure}[htbp]
        \centering
        \includegraphics[width=16cm]{pics/graph.png}
\end{figure}


\section{主要设计点}

\subsection{TLB模块设计}

TLB模块的设计较为简单。读和写可以仿照寄存器堆类似地实现。对于查找,我们先算出match信号,再选择对应的index和tlb表项即可。对于invtlb指令,可以按照教材将逻辑拆分成4组cond[i]的组合,再根据invtlb\_op来设置clr的每一位,最后根据clr将相应的tlb\_e拉低。

\subsection{TLB相关的控制状态寄存器设计}

本次实验中,除了常规的CSR读写指令之外,TLBRD、TLBSRCH也会修改相应的CSR。需要注意的是TLBRD会同时修改多个CSR,原有的数据通路难以直接复用,需要额外添加接口。

此外要注意补完CRMD的DA、PG、datf、datm域的逻辑。因为之前的cpu没有虚实地址转换的功能,DA和PG直接设为了固定值,在本次实验中需要修改。除了常规的读写之外,在初始化、触发TLB重填异常和ertn\_flush时也要按照指令手册的要求设置DA和PG域。BADV的修改条件也要扩展,加入PIS、PIF、PIL、PME、PPI、TLBR例外的出错虚地址。

\subsection{TLB相关指令的添加}

TLB读写指令在WB级实现,主要内容是按照指令手册添加数据通路,完成TLB数据和CSR的交互。添加了一个自增计数器tlb\_fill\_idx用于TLBFILL的随机填充。

TLBSRCH在EX级实现,复用了访存地址转换的数据通路。INVTLB也在EX级实现,除了设置invtlb\_op和invtlb\_valid之外,还要复用s1\_asid和s1\_vppn。

\subsection{MMU设计与TLB相关异常支持}

MMU的功能是接收虚拟地址,根据当前的地址翻译模式决定是否查找DMW和TLB,最后返回物理地址和异常信号,整个模块完全使用组合逻辑实现。我们在mycpu\_top中实例化了两个MMU模块,分别用于取指级和访存级的虚实地址转换。MMU产生的异常信号返回cpu随流水级传递,在写回级根据指令手册规定的优先级进行选择。

\section{调试}

\subsection{s1\_asid信号多驱动}

data\_mmu和EX级都向外输出了s1\_asid信号导致报错,因为tlbsrch指令的的asid要从寄存器得到,所以选择在EX级将地址转换和tlb指令的asid合并后输出。

\subsection{开始时触发指令不存在例外}

又一次遇到了这个问题,因为我们的设计是将next\_pc作为IF级虚地址进行转换,所以之前将初始PC设为32'h1c000000的方式不能继续沿用。因此我们选择把复位时的状态从$ wait\_addr\_ok$改为$readygo$,PC为32'h1bfffffc时不再发出取指请求。

\subsection{tlb重填例外入口地址错误}

触发tlb重填例外时的例外入口地址应设为为TLBRENTRY里存放的的值。注意不要与重取指令的target混淆。

\subsection{TLB相关例外不互斥}

一开始设计的TLB相关例外判断条件不互斥,只通过优先级进行选择,导致部分CSR设置出现错误。修改后ppi、pme、pis/pif/pil、tlbr的判断条件彼此互斥,问题解决。

\subsection{MMU使能设置问题}

在触发例外时没有及时拉低mmu的使能。

\end{document}