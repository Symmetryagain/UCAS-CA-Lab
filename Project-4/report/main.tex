\documentclass{article}

\usepackage[a4paper]{geometry}
\geometry{left=2.5cm,right=2.5cm,top=3cm,bottom=3cm}
\usepackage[UTF8,space,hyperref]{ctex}
\usepackage{amsmath, amsthm, amssymb, bm, color, framed, graphicx, hyperref, mathrsfs, physics}
\hypersetup{hidelinks,colorlinks=true,allcolors=black,pdfstartview=Fit,breaklinks=true}

\title{Project 4 实验报告}
\author{钱翰林 周远扬 石曜铭}
\date{11/11/2025}

\begin{document}
\maketitle

\section{小组分工情况说明}
\begin{itemize}
        \item 钱翰林: CSR 内部实现,调试
        \item 周远扬: 中断,CSR 指令译码与实现
        \item 石曜铭: CSR 指令的实现,调试流水线
\end{itemize}

\section{处理器结构设计图}
\begin{figure}[htbp]
        \centering
        \includegraphics[width=16cm]{pics/graph.png}
\end{figure}


\section{主要设计点}

\subsection{CSR 内部实现}
首先按照指令手册 $7.1$ 节控制状态寄存器一览表去定义 $\texttt{csr\_num}$,并对照手册对各个控制状态寄存器进行分段。
然后阅读教材,参考书上的代码完成 CSR 各个域的更新逻辑,
其中 ECFG、EENTRY、SAVE$0\sim 3$、TID、TCFG 仅会被 CSR 读写指令更新;
PRMD、ERA、BRA在触发异常时也会被更新;
CRMD 在触发异常和 ertn 指令执行时也会被更新;
ESTAT 的各个域的更新逻辑存在区别,需要一一设计;
定时器计数器TVAL在软件开启 timer 的使能时将 $\texttt{csr\_ tcfg\_initval}$ 更新到 $\texttt{timer\_cnt}$ 中,
软件关闭timer的使能时,$\texttt{timer\_cnt}$ 不更新。最后每个CSR重新拼合出一个32位宽的值,
根据 $\texttt{csr\_num}$ 进行选择并返回。

\subsection{CSR 相关指令}

本实验在流水线中加入了CSR相关指令的支持,包括csrrd、csrwr和csrxchg,不同指令对应不同的信号,传入csr分别完成相关读写操作,并且在指令期间阻塞,保证异常安全。


\subsection{各异常与中断}

在实验中,完成了取指地址错(ADEF)、地址非对齐(ALE)、断点(BRK)、系统调用(SYS)和指令不存在(INE)异常以及各种中断。对于异常,分别来自流水线的各个阶段或者指令,将异常信号在WB阶段传给整个流水线,并且拉低各阶段的valid,重新在IF取新指令完成异常处理,另外实现ertn指令完成异常恢复,流程类似。

\section{调试}

\subsection{误将复位后的全零指令视为不存在}
解决方案:对 IF 阶段添加 $\texttt{if\_to\_id\_valid}$ 信号,只有当信号拉高时,才会启动对指令不存在的判断。

\subsection{EX 阶段状态转移不完备}
\begin{itemize}
\item 原先对 EX 阶段状态转移的设计中,默认了 ID 阶段是单周期,当添加 CSR 指令后,ID 阶段可能会有阻塞,导致与 EX 阶段期望的状态不符。

解决方案:在 EX 的状态机中添加与 ID 的握手信号(在代码实现中表现为 EX 状态的 $valid$ 信号),仅当 $valid$ 拉高时才进行状态的转移。

\item 没考虑到 EX 在中间两个等待状态的过程中,被 $flush$ 的情况。

解决方案:在被 $flush$ 后统一转移到 $readygo$ 状态,这是符合逻辑的。
\end{itemize}

\subsection{CSR 相关指令引起的数据冲突解决不完善}
在 rdcntid, rdcntvl.w, rdcntvh.w 指令中,rdcntid 需要读 TID,而结果在 WB 才能得到,所以也不能通过前递解决,需要在 ID 阶段阻塞。
对于 rdcntvl.w, rdcntvh.w 指令,结果在 EX 中得到,因此可以添加进 EX 阶段的 $\texttt{compute\_result}$ 中,跟随原有的数据前递即可。

\textbf{我们注意到,}在通过 exp11 的代码中,EX 阶段的前递数据直接连在了 $\texttt{alu\_result}$ 而不是更完全的 $\texttt{compute\_result}$ 中,
理论上,如果有一条乘除法指令与下一条指令有数据冲突,我们之前设计的处理器行为是不正确的,因为并没有把结果前递。但这仍然能通过 exp11 的测试,可能是测试中不包含针对这种情形的样例所致。

\end{document}